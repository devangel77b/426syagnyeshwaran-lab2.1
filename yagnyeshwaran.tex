\documentclass[reprint,amsmath,amssymb,aps]{revtex4-2}


\usepackage{graphicx}
\usepackage{amsmath,amssymb,amsfonts}
\usepackage{dcolumn}
\usepackage{bm}
\usepackage{siunitx}
\sisetup{separate-uncertainty=true}
\usepackage[colorlinks,allcolors=blue]{hyperref}
\usepackage{cleveref}
\crefname{equation}{}{}
\crefname{figure}{Fig.}{Figs.}
\crefname{table}{Table}{Tables}
\usepackage{svg}




\begin{document}

\title{Verifying $\sum\vec{F} = m\vec{a}$}

\author{Sagarika Yagnyeshwaran}
\email{Contact author: 426syagnyeshwaran@frhsd.com}
\author{Emily Chen}
\author{Andrew Kabatsky}
\author{Alexandra Guimaraes}
\author{Vijita Ayyangar}
\author{Aryanna Cetrulo}
\affiliation{Science \& Engineering Magnet Program, \href{https://manalapan.frhsd.com/}{Manalapan High School}, Englishtown, NJ 07726 USA}
\date{\today}

\begin{abstract}
Newton’s second law is verified in this lab through the use of a cart and rail attached to a pulley system, tested with various masses on the cart and a constant hanging mass. Thus, if we know the force applied and the mass, we can use this equation to calculate the acceleration of an object. Conversely, if we can measure the acceleration and mass of an object, we can also compute the applied force. Our results were confirmed by the values obtained as a result of this experiment.
\end{abstract}

\keywords{keywords here}

\maketitle





\section{Introduction}
Net force is the product of mass times acceleration \cite{newton1687principia}:
\begin{equation} 
\sum \vec{F} = m\vec{a}
\label{eq:1}
\end{equation}
where $\vec{F}$ is the force produced in newtons (\unit{\newton}), $m$ is the mass in kilograms (\unit{\kilo\gram}), and $\vec{a}$ is the acceleration produced in meters per second squared (\unit{\meter\per\second\squared}) \cite{newton1687principia}. \Cref{eq:1} reflects that the force acting on an object involves both its mass and acceleration. Formally, force and acceleration are vector quantities; here we consider one-dimensional (1D) movements and will drop the vector notation for simplicity. 

By applying Newton's second law, we can understand how different forces (gravity, friction, tension, etc.) affect motion. Understanding the relationship between force, mass, and acceleration lets us predict how objects will move under different forces. This helps engineers design systems with precise control over motion, ensuring the stability and safety of structures and vehicles.

Therefore, we wish to verify Newton's second law, $F = ma$. We hypothesize that an applied force on a cart system, of known mass, will produce an acceleration proportional to this force, assuming friction is negligible:
\begin{equation} 
H_0: F = ma 
\label{eq:2}
\end{equation}

Alternatively, if $F$ is not directly proportional to $ma$, we may observe deviations:
\begin{equation} 
H_1: F \neq ma
\label{eq:3}
\end{equation}
We tested these hypotheses by conducting numerous trials and recording the time taken for the cart to travel \qty{0.7}{\meter} from rest as the mass of the mass on the cart changed.







\section{Methods and materials}

\subsection{Finding acceleration}
In the experiment, we weighed each of the masses on a spring scale (Learning Resources; Vernon Hills, IL), with each of the additional masses on the $m_c=\qty{0.5}{\kilo\gram}$ cart being \qty{1.2}{\kilo\gram}, and the smaller hanging mass tied to the string being $m_2=\qty{0.2}{\kilo\gram}$. We tied the smaller mass to a string and strung it across a pulley system, with the other end of the string tied to the cart. For each trial, the mass on the cart which was placed on top of a rail (PASCO Scientific; Roseville, CA) was increased and the time it took for the cart to travel \qty{0.7}{\meter} was recorded. All trials started from rest.
\begin{figure} %[H]
\centering
\includegraphics[width=0.7\linewidth]{Lab Setup Freebody diagram (3).png}
\caption{\label{fig:enter-label} Free body diagram of the system}
\end{figure}


We used a mass pulley system, with a hanging mass of \qty{0.2}{\kilo\gram}, with different masses on the cart (in addition to the cart's mass), starting from: \qtylist{1.2;2.4;3.6}{\kilo\gram}. We measured three different times for each trial, observing how fast it took the cart to travel a distance of \qty{0.7}{\meter} so we could calculate the system's acceleration with different masses $m_1$ as the independent variable. For each trial, we used an iPhone 14 (Apple; Cupertino, CA) to take videos to obtain timing.
\begin{figure} %[H]
\centering
\includegraphics[width=0.4\textwidth]{Fixed_lab_pix3.png}
\caption{\label{fig:lab_setup} Experimental setup showing the cart ($m_c=\qty{0.5}{\kilo\gram}$) system with additional mass $m_1$, string, pulley, and hanging mass $m_2=\qty{0.2}{\kilo\gram}$ pulling the cart via the string. Pull distance was \qty{0.7}{\meter}.}
\end{figure} 

To calculate the measured acceleration of each individual trial for the respective masses, we assumed uniform acceleration and re-arranged the kinematic equation for position, solving for the acceleration $a$ given the time $t$ to move a distance $d$ \cite{tipler}:
\begin{equation} 
a_{meas} = \frac{2d}{t^2}.
\label{eq:4} 
\end{equation}
%This equation was used to calculate the accelerations of each individual trial for the respective masses. \cite{tipler}

We compared the measured acceleration to the acceleration predicted using the masses of the system and the free body diagram given in \cref{fig:enter-label}, a half-Atwood machine configuration with solution commonly available in textbooks \cite{tipler}:
\begin{equation} 
a = \dfrac{m_2}{m_1 + m_2 + m_c} g
\label{eq:5}
\end{equation}
where the denominator indicates the total system mass is given by $m_1+m_2+m_c$ and $g=\qty{9.81}{\meter\per\second\squared}$ is the acceleration of gravity. 
%This equation was used to calculate the predicted acceleration using the masses of the system. \cite{tipler} 

%The standard equations were used to find the mean and standard deviation for the times and accelerations in each trial.








\section{Results}
We measured the time taken for a cart to travel a set distance of \qty{0.7}{\meter}. For each mass $m_1$, three trials were conducted. \Cref{tab:newtable1} gives the measured time and resulting acceleration, calculated using \cref{eq:5}, listed as mean $\pm$ one standard deviation, for $m_1=\qtylist{1.2;2.4;3.6}{\kilo\gram}$. For these measurements, $m_2=\qty{0.2}{\kilo\gram}$ and the empty cart $m_c=\qty{0.5}{\kilo\gram}$. For each value of $m_1$ there were $n=3$ replicates. 
%The data presented in Table (3) represents the expected accelerations which were calculated through  \cref{eq:5} and the expected forces which were calculated using \cref{eq:1}. Furthermore,  Table (2) provides the actual accelerations of the system for each trial which were calculated using \cref{eq:5} as well as the mean calculated accelerations and their standard deviations.
% latex table generated in R 4.4.2 by xtable 1.8-4 package
% Sat Nov 30 17:52:20 2024
\begin{table}[hb]
\caption{\label{tab:newtable1} Measured time $t$ (\unit{\second}) and resulting acceleration $a$ (\unit{\meter\per\second\squared}), listed as mean $\pm$ one standard deviation, for $m_1=\qtylist{1.2;2.4;3.6}{\kilo\gram}$. For these measurements, $d=\qty{0.7}{\meter}$, $m_2=\qty{0.2}{\kilo\gram}$ and the empty cart $m_c=\qty{0.5}{\kilo\gram}$. For each value of $m_1$ there were $n=3$ replicates. }
\begin{center}
\begin{ruledtabular}
\begin{tabular}{ccc}
$m_1$ (\unit{\kilo\gram}) & $t$ (\unit{\second}) & $a$ (\unit{\meter\per\second\squared}) \\ 
\colrule
\num{1.200} & \num{1.03\pm0.08} & \num{1.33\pm0.22} \\ 
\num{2.400} & \num{1.43\pm0.08} & \num{0.69\pm0.07} \\ 
\num{3.600} & \num{1.61\pm0.04} & \num{0.54\pm0.03} \\ 
\end{tabular}
\end{ruledtabular}
\end{center}
\end{table}


%\begin{table} %[H]
%\centering
%%\begin{tabular}{ | m{1.2cm} | m{1.1cm}| m{1.1cm} | m{1.1cm} | m{1cm} |} 
%\begin{tabular}{ccccc}
%%  \hline
%  Mass Added to Cart & Time 1 (s) & Time 2 (s) & Time 3 (s) & Mean + SD \\ 
%%  \hline
%  1.2 kg& 0.94 & 1.08 & 1.08 & 1.03 $\pm$ 0.081\\ 
%%  \hline
%  2.4 kg& 1.43 & 1.36 & 1.51 & 1.43 $\pm$ 0.081 \\ 
%%  \hline
%  3.6 kg& 1.64 & 1.57 & 1.63 & 1.61 $\pm$ 0.042 \\ 
%%  \hline
%\end{tabular}
%\caption{Measured Times for Each Trial} 
%\label{Times of Trials} 
%\end{table}
%
%\begin{table} %[H] 
%\centering
%%\begin{tabular}{ | m{1.2cm} | m{1.2cm}| m{1.2cm} | m{1.2cm} | m{1cm} |} 
%\begin{tabular}{ccccc}
%%  \hline
%  Mass Added to Cart & a(1) in $m/s^2$ & a(2) in $m/s^2$ & a(3) in $m/s^2$ & Mean + SD  \\ 
%%  \hline
%  1.2 kg& 1.584 & 1.200  & 1.200 & 1.33 $\pm$ 0.181\\ 
%%  \hline
%  2.4 kg& 0.746 & 0.657 & 0.614 & 0.67 $\pm$ 0.055 \\ 
%%  \hline
%  3.6 kg& 0.521 & 0.568 & 0.527 & 0.54 $\pm$ 0.021 \\ 
%%  \hline
%\end{tabular}
%\caption{Actual Accelerations ($a$) for Each Trial}
%\label{Accelerations for Each Trial} 
%\end{table}
%
%\begin{table} %[H] 
%\centering
%%\begin{tabular}{ | m{1.2cm} | m{2.1cm}| m{1.2cm}|} 
%\begin{tabular}{cccc}
%%  \hline
%  Trials & Acceleration $m/s^2$ & Force $N$ \\ 
%%  \hline
%  Trial 1 & 1.033 & 1.756\\ 
%%  \hline
%  Trial 2 & 0.633 & 1.836 \\ 
%%  \hline
%  Trial 3 & 0.426 & 1.870 \\ 
%%  \hline
%\end{tabular}
%\caption{ Expected Accelerations (\unit{\meter\per\second\squared}) and Forces (\unit{\newton}) for Each Trial}
%\label{Accelerations for Each Trial} 
%\end{table}

\Cref{fig:graph of acceleration} illustrates the behavior of the expected and calculated accelerations in terms of mass. Measured values of acceleration, calculated from $t$ using \cref{eq:5} and tabulated in \ref{tab:newtable1}, are plotted in \cref{fig:graph of acceleration} as black dots. Acceleration predicted using \cref{eq:5} is plotted as a blue line. The actual acceleration values follow the behavior of the predicted values closely, verifying Newton's second law. As the mass added to the cart increased, we observed a decrease in acceleration, while the force ($T$ in \cref{fig:enter-label}) remained approximately the same. Specifically, the calculated accelerations were \qtylist{1.33;0.67;0.54}{\meter\per\second\squared}, and the forces were \qtylist{1.756;1.836;1.870}{\newton} for masses of \qtylist{1.7;2.9;4.1}{\kilo\gram} (accounting for the cart and the added masses, i.e. $m_1+m_c$) respectively. 
%The hanging mass of \qty{0.2}{\kilo\gram} was kept constant throughout all the trials; i.e. external gravitational force was held constant through our experiment. 

\begin{figure}[ht] %[H]
\begin{center}
%\includegraphics[width=0.8\linewidth]{desmos-graph (2).png}
\includesvg[width=\columnwidth]{fig3.svg}
\end{center}
\caption{\label{fig:graph of acceleration} Mass $m_1$ (\unit{\kilo\gram}) versus acceleration $a$ (\unit{\meter\per\second\squared}). Dots indicate measured values of acceleration obtained from the measured time to travel from rest \qty{0.7}{\meter} using \cref{eq:4}. $n=3$ replicates for each value of $m_1$, as tabulated in \cref{tab:newtable1}. Blue line indicates acceleration predicted by \cref{eq:5} for $m_1$, $m_2=\qty{0.200}{\kilo\gram}$, and empty cart $m_c=\qty{0.500}{\kilo\gram}$.}
\end{figure}

For each of the trials, the standard deviations of the accelerations were minimal (\cref{tab:newtable1}), indicating precise results. 





\section{Discussion}
Our results support Newton's second law (\cref{eq:1}), showing the acceleration was inversely proportional to the total mass of the cart. As shown in \cref{fig:graph of acceleration}, measured acceleration values align with acceleration predicted from Newton's second law (\cref{eq:1}) based on analysis of the free-body diagram of \cref{fig:enter-label}, corroborating that mass and acceleration are inversely proportional to each other when force is kept the same. Our test was conducted with varying total system mass and constant force; however, if mass remains the same, acceleration and force will be directly proportional. 

Ultimately, force calculations should be consistent as the mass increases, but we observed a slight discrepancy from our results compared to the expected values, likely due to human error in timing with stopwatches. As a result of minor errors during timing, the calculated acceleration values were slightly higher than the predicted acceleration values. This is visible in \cref{fig:graph of acceleration}, where most of the measured acceleration values from the experiment are plotted slightly higher than the curve of the predicted acceleration. Though the plane was near frictionless, the little friction should have decreased the acceleration in comparison to the expected values; however, our human error was significant and offset the difference caused by friction. 







\section{Acknowledgements}
We thank A Ortega for her momentum writeup which served as a reference. Additionally, we thank several anonymous reviewers whose comments improved our manuscript.

EC, AG, AK, and SY developed  the first draft of the manuscript. VA, AC, AG, and AK collected the data. Everyone contributed to revisions.






%\begin{thebibliography}{9}
%\bibitem{ref1} I. Newton, \textit{Philosophiæ Naturalis Principia Mathematica}, S. Pepys, London, 1687.
%\bibitem{ref2} P. Tipler, G. Mosca, \textit{Physics for Scientists and Engineers, Extended}, W.H. Freeman and Company, United States, 2004.
%\end{thebibliography}
%\bibliographystyle{abbrvnat}
\bibliography{lab.bib}
\end{document}
