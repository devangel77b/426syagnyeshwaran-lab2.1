\documentclass[twocolumn, 10pt]{article}
\usepackage{amsmath}
\usepackage{graphicx}
\usepackage{caption}
\usepackage{float}
\usepackage{multicol}
\usepackage{array}

% Title and Author section formatting
\title{\textbf{Verifying $\Sigma\vec{F} = m\vec{a}$}} 

\author{
     S. Yagnyeshwaran, E. Chen, A. Kabatsky, A. Guimaraes, V. Ayyangar, A. Cetrulo
}
\date{November 15, 2024} % No date

% Page layout
\setlength{\textwidth}{6.5in}
\setlength{\oddsidemargin}{0in}
\setlength{\evensidemargin}{0in}
\setlength{\topmargin}{-0.5in}
\setlength{\textheight}{9in}
\setlength{\parindent}{0.5cm}

\begin{document}
\twocolumn[
\begin{@twocolumnfalse}
    \maketitle
    \centering
    \begin{abstract}
    Newton’s Second Law is verified in this lab through the use of a cart and rail attached to a pulley system, tested with various masses on the cart and a constant hanging mass. Thus, if we know the force applied and the mass, we can use this equation to calculate the acceleration of an object. Conversely, if we can measure the acceleration and mass of an object, we can also compute the applied force. Our results were confirmed by the values obtained as a result of this experiment.
    \end{abstract}
    \vspace{0.5cm}
\end{@twocolumnfalse}
]

\section*{INTRODUCTION}
Net force is the product of mass times acceleration:
\begin{equation} \label{eq:1}
    \Sigma\vec{F} = m\vec{a} \cite{ref1}
\end{equation}
where $\vec{F}$ is the force produced in Newtons (N), $m$ is the mass in kilograms (kg), and $\vec{a}$ is the acceleration produced in meters per second squared ($m/s^2$) \cite{ref1}. Equation (\ref{eq:1}) reflects that the force acting on an object involves both its mass and acceleration. Formally, force and acceleration are vector quantities; here we consider one-dimensional (1D) movements and will drop the vector notation for simplicity. 

By applying Newton's Second Law, we can understand how different forces (gravity, friction, tension, etc.) affect motion. Understanding the relationship between force, mass, and acceleration lets us predict how objects will move under different forces. This helps engineers design systems with precise control over motion, ensuring the stability and safety of structures and vehicles.

Therefore, we wish to verify Newton's Second Law, $F = ma$. We hypothesize that an applied force on a cart system, of known mass, will produce an acceleration proportional to this force, assuming friction is negligible:
\begin{equation} \label{eq:2}
    H_0: F = ma 
\end{equation}

Alternatively, if \( F \) is not directly proportional to \( ma \), we may observe deviations:
\begin{equation} \label{eq:3}
    H_1: F \neq ma
\end{equation}
We tested this hypothesis by conducting numerous trials and recording the time taken for the cart to travel the set distance of 0.7 m as the mass of the mass on the cart changed
\section*{METHODS AND MATERIALS}
\subsection*{Finding Acceleration}
In the experiment, we weighed each of the masses on a spring scale (Learning Resources, IL), with each of the additional masses on the 0.5 kg cart being 1.2 kg, and the smaller hanging mass tied to the string being 0.2 kg (Manalapan, NJ). We tied the smaller mass to a string and strung it across a pulley system, with the other end of the string tied to the cart. For each trial, the mass on the cart which was placed on top of a rail(PASCO Scientific; Roseville, CA) was increased and the time it took for the cart to travel 0.7 m was recorded.

\begin{figure} [H]
    \centering
    \includegraphics[width=0.7\linewidth]{Lab Setup Freebody diagram (3).png}
    \caption{Freebody Diagram of System}
    \label{fig:enter-label}
\end{figure}


We used a mass pulley system, with a hanging mass of 0.2 kg, with different masses on the cart (in addition to the cart's mass), starting from: 1.2 kg, 2.4 kg, and 3.6 kg. We measured three different times for each trial, observing how fast it took the cart to travel a distance of 0.7 m so we could calculate the system's acceleration with different masses. For each trial, we used an iPhone 14 (Apple; Cupertino, CA). to take videos.
\begin{figure}[H]
    \centering
    \includegraphics[width=0.4\textwidth]{Fixed_lab_pix3.png}
    \caption{Lab Setup}
    \label{fig:lab_setup}
\end{figure} 

\begin{equation} \label{eq:4}
    a = \frac{2d}{t^2} 
\end{equation}
This equation was used to calculate the accelerations of each individual trial for the respective masses. \cite{ref2}

\begin{equation} \label{eq:5}
    a = \frac{m_2}{m_1+m_2} * g
\end{equation}

This equation was used to calculate the predicted acceleration using the masses of the system. \cite{ref2} 
\newline \indent

The standard equations were used to find the mean and standard deviation for the times and accelerations in each trial.

\section*{RESULTS}


\begin{table}[H]
\centering
\begin{tabular}{ | m{1.2cm} | m{1.1cm}| m{1.1cm} | m{1.1cm} | m{1cm} |} 
  \hline
  Mass Added to Cart & Time 1 (s) & Time 2 (s) & Time 3 (s) & Mean + SD \\ 
  \hline
  1.2 kg& 0.94 & 1.08 & 1.08 & 1.03 $\pm$ 0.081\\ 
  \hline
  2.4 kg& 1.43 & 1.36 & 1.51 & 1.43 $\pm$ 0.081 \\ 
  \hline
  3.6 kg& 1.64 & 1.57 & 1.63 & 1.61 $\pm$ 0.042 \\ 
  \hline
\end{tabular}
\caption{Measured Times for Each Trial} 
    \label{Times of Trials} 
\end{table}

\begin{table}[H] 
\centering
\begin{tabular}{ | m{1.2cm} | m{1.2cm}| m{1.2cm} | m{1.2cm} | m{1cm} |} 
  \hline
  Mass Added to Cart & a(1) in $m/s^2$ & a(2) in $m/s^2$ & a(3) in $m/s^2$ & Mean + SD  \\ 
  \hline
  1.2 kg& 1.584 & 1.200  & 1.200 & 1.33 $\pm$ 0.181\\ 
  \hline
  2.4 kg& 0.746 & 0.657 & 0.614 & 0.67 $\pm$ 0.055 \\ 
  \hline
  3.6 kg& 0.521 & 0.568 & 0.527 & 0.54 $\pm$ 0.021 \\ 
  \hline
\end{tabular}
\caption{Actual Accelerations (a) for Each Trial}
    \label{Accelerations for Each Trial} 
\end{table}

\begin{table}[H] 
\centering
\begin{tabular}{ | m{1.2cm} | m{2.1cm}| m{1.2cm}|} 
  \hline
  Trials & Acceleration $m/s^2$ & Force $N$ \\ 
  \hline
  Trial 1 & 1.033 & 1.756\\ 
  \hline
  Trial 2 & 0.633 & 1.836 \\ 
  \hline
  Trial 3 & 0.426 & 1.870 \\ 
  \hline
\end{tabular}
\caption{ Expected Accelerations ($m/s^2$) and Forces (N) for Each Trial}
    \label{Accelerations for Each Trial} 
\end{table}

\begin{figure} [H]
    \centering
    \includegraphics[width=0.8\linewidth]{desmos-graph (2).png}
    \caption{Mass (kg) vs. Acceleration ($m/s^2$)}
    \label{fig:graph of acceleration}
\end{figure}


 \indent We measured the time taken for a cart to travel a set distance of 0.7 m. For each mass, three trials were conducted. For each trial, the mean time and standard deviations of time were calculated using the standard equations. This information is presented in Table (1). The data presented in Table (3) represents the expected accelerations which were calculated through equation (\ref{eq:5}) and the expected forces which were calculated using equation (\ref{eq:1}). Furthermore,  Table (2) provides the actual accelerations of the system for each trial which were calculated using equation (\ref{eq:5}) as well as the mean calculated accelerations and their standard deviations.
\newline \indent FIG (\ref{fig:graph of acceleration}) illustrates the behavior of the expected and calculated accelerations in terms of mass. The actual values are plotted on the graph in black and the predicted acceleration that was graphed using the equation (\ref{eq:4}) is graphed in purple. The actual acceleration values follow the behavior of the predicted values very closely, verifying Newton's Second Law. As the mass added to the cart increased, we observed a decrease in acceleration, while the force remained approximately the same. Specifically, the calculated accelerations were 1.33 $m/s^2$, 0.67 $m/s^2$, and 0.54 $m/s^2$, and the forces were 1.756 $N$, 1.836 $N$, and 1.870 $N$ for masses of 1.7 kg, 2.9 kg, and 4.1 kg (accounting for the cart and the added masses)  respectively. The hanging mass of 0.2 kg was kept constant throughout all the trials. 
\newline \indent For each of the trials, the standard deviations of the accelerations were minimal (Table 2), indicating fairly accurate results. The predicted values align with the calculated values, corroborating that mass and acceleration are inversely proportional to each other when force is kept the same. Since this relationship is verified, if mass remains the same,  acceleration and force will be directly proportional. 

\section*{DISCUSSION}
 \indent The results do support Newton's Second Law, showing the acceleration was inversely proportional to the total mass of the cart. Ultimately, force calculations should be consistent as the mass increases, but we observed a slight discrepancy from our results compared to the expected values, likely due to human error. As a result of minor errors during timing, the calculated acceleration values were slightly higher than the predicted acceleration values. This is visible in the graph, where most of the acceleration values from the experiment are plotted slightly higher than the curve of the predicted acceleration. Though the plane was near frictionless, the little friction should have decreased the acceleration in comparison to the expected values; however, our human error was significant and offset the difference caused by friction. 

\section*{ACKNOWLEDGMENTS} 
We thank Antonella Ortega for her momentum writeup which served as a reference. Additionally, we acknowledge each of the members of our group and their contributions. Thank you to E. Chen, A. Guimaraes, A. Kabatsky, and S. Yagnyeshwaran for your hard work on the first draft of the lab report. Thank you to V. Ayyangar, A. Cetrulo, A. Guimaraes, and A. Kabatsky for their data collection. Everyone contributed to the revisions made to this writeup. Lastly, we would like to thank D. Avalur, C. Butash, C. Canada, D. Evangelista, and K. Tomazic for providing suggestions and critiques on our work. 

\begin{thebibliography}{9}
\bibitem{ref1} I. Newton, \textit{Philosophiæ Naturalis Principia Mathematica}, S. Pepys, London, 1687.
\bibitem{ref2} P. Tipler, G. Mosca, \textit{Physics for Scientists and Engineers, Extended}, W.H. Freeman and
Company, United States, 2004.
\end{thebibliography}
\end{document}
